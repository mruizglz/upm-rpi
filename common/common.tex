%\documentclass[fleqn,xcolor=dvipsnames]{beamer}
\usetheme{Madrid}
%\usetheme{Boadilla}
%\usetheme{default}
%\usetheme{Warsaw}
%\usetheme{CambridgeUS}
%\usetheme{Sybila}
%\usetheme[hideothersubsections]{Berkeley}
%\usetheme[hideothersubsections]{PaloAlto}
%%\usetheme[hideothersubsections]{Goettingen}
%\usetheme{CambridgeUS}
%\usetheme{Bergen} % This template has nagivation on the left
%\usetheme{Frankfurt} % Similar to the default with an extra region at the top.

% Uncomment the following line if you want page numbers and using Warsaw theme%
%\setbeamertemplate{footline}[page number]

\definecolor{aggiemaroon}{RGB}{80,0,0} % Official RGB code for aggie maroon
\usecolortheme[named=aggiemaroon]{structure}
\useoutertheme{shadow}
\useinnertheme{rounded}
\setbeamertemplate{navigation symbols}{}
\setbeamerfont{structure}{family=\rmfamily,series=\bfseries}
\setbeamerfont{footline}{size=\tiny}
\usefonttheme[stillsansseriftext]{serif}

\usepackage{amsmath}
\usepackage{caption}
\usepackage{graphicx,comment}
\usepackage[english]{babel}
\usepackage{rotating}
\usepackage{multicol}
\usepackage{enumerate}
%\usepackage{enumitem}
\usepackage{tikz}
\usepackage{bm}
\usepackage{minted}

\usepackage{xcolor}
\usepackage{subcaption}
\usepackage{listings}
\usepackage{hyperref} %always the last to avoid problems


\usebackgroundtemplate{
\tikz\node[opacity=0.035, rotate = 0] {\includegraphics[scale=0.5]{UPM-logo.png}};}
%{\includegraphics[height=1in,width=1in]{TAM-Logo.png}};}


\title[Universidad Politecnica de Madrid]{Embedded System Design with Raspberry-pi}

\author [M. Ruiz]{Mariano Ruiz}

\date[01/15/2025]{February-May 2025}

\institute[UPM] % (optional, but mostly needed)
{\emph{Professor}, ETSI Sistemas de Telecomunicación Copiado de Bootlin \\
 \includegraphics[scale=0.60]{UPM-logo.png} \\ [0.0 cm]
 }



\AtBeginSection[]
{
  \begin{frame}<beamer>
   % \frametitle{Section \thesection} %this add the section number only
   \frametitle{\insertsectionhead}
    \tableofcontents[currentsection, hideallsubsections]
  \end{frame}
}
%\newcommand{\codelink}[1]{\texttt{#1}}

\newcommand{\codecolor}{\usebeamercolor[fg]{code}}
\newcommand{\code}[1]{{\codecolor \path{#1}}}
\newcommand{\codelink}[1]{{\usebeamercolor[fg]{darkblue} \path{#1}}}
\newcommand{\codewithhash}[1]{{\codecolor \tt{#1}}}
%\newcommand{\code}[1] {\path{#1}}

\newcommand\projdir[2]{\href{https://elixir.bootlin.com/#1/latest/source/#2/}{\codelink{#2/}}}
\newcommand\projfile[2]{\href{https://elixir.bootlin.com/#1/latest/source/#2}{\codelink{#2}}}
\newcommand\kconfig[1]{\href{https://elixir.bootlin.com/linux/latest/K/ident/#1}{\codelink{#1}}}
\newcommand\kconfigval[2]{\href{https://elixir.bootlin.com/linux/latest/K/ident/#1}{\codelink{#1=#2}}}
\newcommand\kdir[1]{\projdir{linux}{#1}}
\newcommand\kfile[1]{\projfile{linux}{#1}}
\newcommand\kfileversion[2]{\href{https://elixir.bootlin.com/linux/v#2/source/#1}{\codelink{#1}}}
\newcommand\kstruct[1]{\href{https://elixir.bootlin.com/linux/latest/ident/#1}{\codelink{struct #1}}}
\newcommand\kdochtml[1]{\href{https://www.kernel.org/doc/html/latest/#1.html}{\codelink{#1}}}
\newcommand\kdochtmldir[1]{\href{https://www.kernel.org/doc/html/latest/#1/}{\codelink{#1/}}}
\newcommand{\training}{linux-kernel}

\lstset{
  language=C,
  frame=single, % Adds a frame around the code
  basicstyle=\ttfamily, % Typewriter font
  keywordstyle=\color{blue}, % Keywords in blue
  commentstyle=\color{green}, % Comments in green
  stringstyle=\color{red}, % Strings in red
  numbers=left, % Line numbers on the left
  numberstyle=\tiny\color{gray}, % Line number style
  breaklines=true, % Line breaking
  %backgroundcolor=\color{lightgray!20} % Background color
}